\begin{myrecipe}
    {Ravioli di ricotta e spinaci}
    {4 Personen}
    {2 Stunden}
    {ravioli_ricotta_header.jpg}
    {Foto von Yoav Aziz auf Unsplash}
    \freeform
    \large Ravioli selbst herstellen und füllen ist zwar ein relativ aufwändiger Prozess, jedoch auch ein fast schon meditativer. Einer der Klassiker unter den Ravioli-Füllungen, Ricotta mit Blattspinat, wird hier vorgestellt.
    \freeform
    \notebox{Der Pastateig wird hier nicht explizit aufgeführt, du kannst das Rezept jedoch auf Seite~\pageref{Pastateig} nachlesen.}
    \newpage

    \freeform
    \lineskip0pt\mbox{}\\[-\baselineskip]%
    \parbox[b]{\textwidth}{%
    \rule{0pt}{\baselineskip}%
    \strut{\recipetitlefont Ravioli di ricotta e spinaci}\strut\hfill}%

    \Ing{250g Blattspinat}
    \Ing{250g Ricotta}
    \Ing{50g Parmesan}
    \Ing{1 Knoblauchzehe}
    \Ing{1 Prise Pfeffer}
    \Ing{1 Prise Salz}
    \Ing{wenig Muskat}
    Spinat kurz blanchieren und kleinhacken. Spinat, Ricotta, Parmesan und die gepresste Knoblauchzehe vermengen, mit Gewürzen abschmecken und danach kühlstellen.
    \Ing{1 Ei}
    In Schüssel verquirlen, wird als Leim verwendet.
    \freeform
    \\Der zuvor vorbereitete Pasta-Teig kann nun auf der Walze oder von hand ausgewallt werden. Die Blätter sollten so dünn ausgewallt sein, dass man die eigene Hand dadurch erkennen kann.
    \\Nun entweder mit dem Raviolistempel oder von Hand die Blätter in Quadrate von ca. 6x6cm schneiden und auf einer Unterlage auslegen.
    \\1 Teelöffel der Füllung auf ein Quadratgeben, Gegenseite mit Eigelb bestreichen, auf die Füllung legen und auf allen Seiten gut andrücken.
    \freeform
    \\Die Ravioli in ausreichend gesalzenem, siedenden Wasser ca. 5 Minuten ziehen lassen.
    \freeform
    \tipbox{Die fertig gekochten Ravioli kurz in geschmolzener Butter schwenken. Noch besser wird das ganze, wenn die Butter mit Salbei oder Thymian/Rosmarin parfümiert wird.}
\end{myrecipe}
\clearpage