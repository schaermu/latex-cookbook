\begin{myrecipe}
    {Pizzateig}
    {4 Personen}
    {2\fr{1}{2} Stunden}
    {pizza_teig_header.jpg}
    {Foto von Juan Manuel Núñez Méndez auf Unsplash}

    \freeform
    \notebox{Anstelle von Hefe kann auch Trockenhefe verwendet werden (ca. 4g) mit dem Vorteil, dass diese nicht angerührt werden muss.}
    \notebox{Die Mehl-/Griess-Mischung kann auch mit Pizzamehl ersetzt werden.}
    \tipbox{Getrockneten Oregano direkt zum Mehl dazugeben und in den Teig einarbeiten, ergibt ein super Aroma im Teig.}
    \newpage

    \freeform
    \lineskip0pt\mbox{}\\[-\baselineskip]%
    \parbox[b]{\textwidth}{%
    \rule{0pt}{\baselineskip}%
    \strut{\recipetitlefont Pizzateig}\strut\hfill}%
    
    \Ing{450g Mehl}
    \Ing{50g Hartweizengriess}
    \Ing{ca. 10g Hefe}
    \Ing{2-3 TL Salz}
    \Ing{2\fr{1}{2} dl lauwarmes Wasser}
    \Ing{3 EL Olivenöl extra vergine}
    Hefe im lauwarmen Wasser auflösen. Das Mehl, den Griess und das Salz in einer Schüssel mischen.
    \\Wasser und Öl in die Schüssel geben und während 10-15 Minuten zu einem weichen und geschmeidigen Teig kneten.
    \freeform
    \\Mit einem warm-feuchten Tuch abdecken und ca. 2h aufgehen lassen.
    \freeform
    \\Die belegte Pizza im vorgeheizten Ofen auf oberster Stufe (falls möglich Unterhitze dazuschalten) zwischen 8 und 12 Minuten kross backen.
    \freeform
    \tipbox{Damit die Pizza extra-knusprig wird, das Backpapier vor dem Auslegen mit Hartweizengriess bestreuen.}
\end{myrecipe}
\clearpage