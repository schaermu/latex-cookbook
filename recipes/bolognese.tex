\begin{myrecipe}
    {Ragù alla Bolognese}
    {4 Personen}
    {3\fr{1}{2} - 4 Stunden}
    {bolognese_header.jpg}
    {Foto von Ketut Subiyanto von Pexels}
    \freeform
    \large Ragù alla Bolognese: toll zu Pasta, wunderbar in der Lasagne, genial als Conchiglione-Füllung. Die hier gezeigte Variante ist eher dickflüssig und dadurch super für Lasagne geeignet.
    \freeform
    \tipbox{Bei den Gewürzen kann experimentiert werden: ich verwende beim Andünsten häufig eine pikante italienische Gewürzmischung und getrockneten Oregano fürs Einkochen. Gerade weil in dieser Variante nicht mit Boullion aufgegossen wird, darf liberal gewürzt werden.}
    \newpage

    \freeform
    \lineskip0pt\mbox{}\\[-\baselineskip]%
    \parbox[b]{\textwidth}{%
    \rule{0pt}{\baselineskip}%
    \strut{\recipetitlefont Ragù alla Bolognese}\strut\hfill}%

    \Ing{100g Pancetta}
    \Ing{1 Sellerie (ca. 150g)}
    \Ing{2 Karotten}
    \Ing{1 grosse Zwiebel}
    Sellerie, Zwiebeln und Karotten (sehr) fein würfeln und Pancetta in kleine Würfel schneiden.

    \Ing{2 EL Rapsöl}
    \Ing{1 Prise grobes Meersalz}
    \Ing{1 Prise Pfeffer}
    Öl in einem Topf erhitzen, Salz und Pfeffer kurz andünsten. Pancetta glasig andünsten. Sobald der Speck ausgelassen ist, Zwiebeln und Sellerie hinzugeben, ein paar Minuten später die Karotten.

    \Ing{300g Rindshack}
    \Ing{200g Schweinegehacktes}
    In separatem Topf Schweinegehacktes scharf anbraten, nach ein paar Minuten Rindsgehacktes hinzugeben. Sobald das Fleisch angebraten ist, alles in den Haupttopf geben.

    \Ing{1 guter EL Tomatenmark}
    \Ing{warmes Wasser}
    \Ing{50ml Vollmilch}
    Tomatenmark, ein bisschen warmes Wasser und ein Schuss der Milch beigeben. Mindestens 2h auf kleinster Flamme köcheln lassen, gelegentlich umrühren und ein wenig Milch dazugeben.
    \freeform
    \tipbox{Optional kann zusätzlich eine Dose Pelati beigefügt werden, so wird die Sauce ein bisschen tomatiger.}
\end{myrecipe}
