\begin{myrecipe}
    {Slow-Cook Tomatensauce}
    {4 Personen}
    {3\fr{1}{2} Stunden}
    {tomatensauce_header.jpg}
    {Foto von Alla Zhuk von Pexels}
    \freeform
    \large Dieses Rezept tritt den Beweis an, dass gute Dinge manchmal wirklich Weile haben wollen. Geschmacklich ist dieses Gericht eine Wucht: langsam gegarte, süssliche San Marzano Tomaten treffen auf Rosmarin- und Thymiandüfte, abgerundet durch eine Knoblauchnote.
    \freeform
    \notebox{Als Pasta empfehlen sich z.B. Spaghetti, Linguine oder auch Tagliatelle, da die Sosse an diesen am besten haftet. Für eine vegetarische Variante kann der Speck auch weggelassen werden.}
    \newpage

    \freeform
    \lineskip0pt\mbox{}\\[-\baselineskip]%
    \parbox[b]{\textwidth}{%
    \rule{0pt}{\baselineskip}%
    \strut{\recipetitlefont Slow-Cook Tomatensauce}\strut\hfill}%

    \Ing{500g San Marzano Tomaten}
    \Ing{1 Knoblauchzehe}
    \Ing{\fr{1}{2} Bund Rosmarin}
    \Ing{\fr{1}{2} Bund Thymian}
    \Ing{1 EL brauner Zucker}
    \Ing{grosse Prise Meersalz}
    Tomaten vierteln, Knoblauch in Scheiben schneiden. Tomaten, Knoblauch, Rosmarin, Thymian, Zucker und Salz in eine kleine Auflaufform geben, mischen und während 3 Stunden bei 80 Grad im Ofen garen.

    \Ing{80g Pancetta}
    \Ing{1 Zwiebel}
    \Ing{1 Peperoncini}
    Pancetta in Streifen schneiden, Zwibel fein würfeln und Peperoncini in feine Streifen schneiden.
    \freeform
    \centering
    \itshape Bei Garende kann das Pastawasser aufgesetzt werden.

    \Ing{500g Pasta}
    Pasta al dente kochen.
    Tomaten aus dem Ofen holen, häuten und würfeln. Die Flüssigkeit mit den Tomaten beiseitestellen.
    \Ing{3 EL Olivenöl}
    \Ing{1 EL Tomatenpüree}
    Zwiebeln und Speck in Öl anbraten. Nach einigen Minuten Peperoncini und gepressten Knoblauch dazugeben, ca. 5 Minuten andünsten. Tomatenpüree beigeben,gut mischen und mitdünsten.
    \\Saft und Tomaten beigeben, warm werden lassen, mit Pfeffer und Salz abschmecken.

    
\end{myrecipe}
