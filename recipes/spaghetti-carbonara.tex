\begin{myrecipe}
    {Spaghetti Carbonara}
    {2 Personen}
    {20 Minuten}
    {carbonara_header.jpg}
    {Foto von Sebastian Coman Photography von Pexels}
    \freeform
    \large Ein Klassiker aus der italienischen Küche: Spaghetti Carbonara. In diesem Rezept wird ursprüngliche Variante vorgestellt, das heisst sie wird ohne Rahm zubereitet.
    \freeform
    \cautionbox{Da Essen Geschmacks- und nicht Glaubenssache sein soll, sei es an dieser Stelle dem Leser freigestellt, das Gericht mit Rahm zu...verfeinern...}
    \newpage

    \freeform
    \lineskip0pt\mbox{}\\[-\baselineskip]%
    \parbox[b]{\textwidth}{%
    \rule{0pt}{\baselineskip}%
    \strut{\recipetitlefont Spaghetti Carbonara}\strut\hfill}%

    \Ing{100g Guanciale/Pancetta}
    \Ing{1 Knoblauchzehe}
    \Ing{\fr{1}{2} Bund Petersilie}
    Speck würfeln, Petersilie waschen und kleinschneiden, Knoblauch schälen und vierteln.
    \Ing{230g Spaghetti}
    Genügend Pastawasser aufsetzen, Pasta al dente kochen.
    \freeform
    \\Ca. 5min nach dem Aufsetzen der Pasta Speck mit Knoblauch bei mittlerer Hitze in einer Bratpfanne auslassen.\\
    \Ing{3 Eier}
    \Ing{Pfeffer}
    \Ing{50g Parmesan}
    1 ganzes Ei und 2 Eigelbe mit dem Pfeffer und der \fr{1}{2} des Parmesans vermischen.
    \freeform
    \\Pasta abschütten, 1-2 Kellen Pastawasser beiseitestellen. Pasta zurück in den Topf, Speck dazugeben (darf auch ein bisschen Fett mitkommen).\\
    Ei-Pfeffer Mischung und ein wenig Pastawasser dazugeben und schwenken, bis die Sosse cremig ist. Wenn das ganze trocken wird, Pastawasser nachgiessen.
\end{myrecipe}
\clearpage